\begin{Solution}{Match Maker}

\begin{frame}{\ProblemName}

\small Problem Author: Liu Hongyang, Problem Developer: Liu Hongyang \par \vspace{0.3cm}

\small Submissions: Div.2 (?+?)/(?+?), Div.1 (?+?)/(?+?)  \par \vspace{0.5cm}

Given $n$ $(n \leq 200000)$ integers up to $10^{12}$, decide whether we can select at least half of these numbers, such that they have common divisor which is greater than 1.

\par

Note that such factor appears in at least half of the numbers, so we may employ randomized algorithm. We may randomly pick a number, factorize it in sqrt time, and for each of its factor, check if it is a factor of at least half of the numbers.

\par

However, the number of factors of an integer up to $10^{12}$ can be large, and it is infeasible to check all of them.

\end{frame}

\begin{frame}{\ProblemName}

Actually, we can only check prime factors, since if a non-prime factor $f$ is valid, then any prime factor of $f$ is also valid. There are at most $\log x$ different prime factors for integer $x$.

\par

The algorithm can find a valid factor with probability of at least $1/2$ in one repetition. If we repeating this process for multiple times, the algorithm will give correct answer with high probability.

\par

The total time complexity is $O(T(\sqrt{x} + n \log x))$, where $T$ is the number of repetitions.

\end{frame}

\end{Solution}
