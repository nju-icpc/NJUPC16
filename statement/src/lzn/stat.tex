\addtocounter{ProblemNo}{1}
\renewcommand{\ProblemName}{数据压缩}
\begin{center}
\huge{\Alph{ProblemNo}. \textbf{\ProblemName}} \\ [0.8cm]
\large{\textit{时间限制:} 2 \textit{秒}} \\ [1cm]
\end{center}

Lempel-Ziv是一类常用的无损压缩算法,其基本思想是利用被压缩串中的重复出现的子串,减少存储所需空间。例如,给定被压缩串001010001010111,当我们完成0010100的压缩时,可以发现,剩下的部分(01010111)的前5位在之前出现过,这样,我们只需要拷贝从当前位置6个字符前开始,长度为5的子串即可。

\begin{figure}[h]
\centering
\begin{BVerbatim}
 |---| 
001010001010111
       ^~~~~
\end{BVerbatim}
\end{figure}

事实上,被拷贝的子串还可以和当前被压缩的串重叠,只要被拷贝的串的起始位置在当前位置之前即可。例如:对于被压缩串1010101,如果已经完成了10的压缩,还需要压缩10101,那么我们可以拷贝从当前位置前第2个字符开始,长度为5的子串即可。这里的拷贝可以理解成逐位拷贝。

\begin{figure}[h]
\centering
\begin{BVerbatim}
|----|
1010101
  ^~~~~
\end{BVerbatim}
\end{figure}

Lempel-Ziv-JYY是一种Lempel-Ziv算法的实现。

为方便起见,我们约定:
\begin{enumerate}
    \item $S$表示长度为$n$的二进制串,其下标从0开始,即$S = S_0 S_1 \cdots S_{n-1}$;
    \item $S[i:j]$表示$S$从下标$i$开始,长度为$j-i$的子串,即$S[i:j] = S_iS_{i+1} \cdots S_{j-1}$。特别的,当$i = j$时,$S[i:j]$表示空串。
\end{enumerate}

\begin{algorithm}
\caption{Lempel-Ziv-JYY 算法}
\begin{algorithmic}[1]
    \Require 待压缩的二进制串 $S$
    \Ensure 压缩的结果,包含待压缩串的首位$S_0$和若干形如\texttt{(prev, len)}的二元组。
    \State 输出二进制串的首位 $S_0$。
    \State $n \leftarrow \mathit{length}(S)$
    \State $\mathit{ptr} \leftarrow 0$
    \Repeat
        \State $\mathit{ptr} \leftarrow \mathit{ptr} + 1$
        \State 找出最大的$x$,使得存在$k < \mathit{ptr}$,满足$S[k:k+x] = S[\mathit{ptr}:\mathit{ptr}+x]$
        \State 输出$(\mathit{ptr} - k, x)$
        \State $\mathit{ptr} \leftarrow \mathit{ptr} + x$
    \Until{$\mathit{ptr} = n$}
\end{algorithmic}
\end{algorithm}

给你上述压缩算法输出的结果,你需要恢复被压缩的二进制串。

\subsection*{输入格式}

第一行包含一个整数$n$和$S[0]$,表示程序第4行输出的对数和被压缩串的第一位。

接下来$n$行,每行包含两个数字$a, b$,表示程序第4行的一个输出。

保证输入合法。保证被压缩的串长不超过$10^6$。

\subsection*{输出格式}

输出一行一个01串,表示答案。

\setcounter{ExampleNo}{0}

\exmpv{01-sample}
\exmpv{02-sample}
\exmpv{03-sample}

\clearpage

\ifodd\value{page}
\else
    \vspace*{\fill}
    \begin{center}
    \textbf{\Large 本页无正文}
    \end{center}
    \vspace*{\fill}
    \clearpage
\fi

