\begin{Problem}{Coffee or Chicken}{\divsel{1}{2}}

JYY创造了一种“咖啡鸡字符串”,类似于“斐波那契例汤”——第$n$天的汤是由第$n-2$天和第$n-1$天的汤混合而成的:

\begin{itemize}
  \item $S(1) = \texttt{"COFFEE"}$;
  \item $S(2) = \texttt{"CHICKEN"}$;
  \item $S(n) = S(n - 2) :: S(n - 1)$,其中“$::$”是字符串的拼接运算符。
\end{itemize}

随着$n$的增长,$S(n)$的长度很快就失控了,JYY的小霸王学习机没有足够的内存存储$S(n)$。JYY希望你帮助已经被降智的他求出$S(n)$从第$k$个位置开始的连续10个字符。

\subsection*{输入格式}

\divsel{

    输入的第一行包含一个整数$T$ $(1 \leq T \leq 1000)$,表示测试用例的组数。

    接下来$T$行,每行包含两个整数 $n$ $(1\le n\le \divsel{500}{30})$, $k$ $(1\le k\le \min\{|S(n)|, \divsel{10^{12}}{10^6}\})$,表示一组测试用例。其中$|S|$表示字符串$S$的长度。

}{

    输入数据第一行包含两个整数 $n$ $(1\le n\le \divsel{500}{30})$,$k$ $(1\le k\le \min\{|S(n)|, \divsel{10^{12}}{10^6}\})$,其中$|S|$表示字符串$S$的长度。

}

\subsection*{输出格式}

\divsel{对于每组测试数据,}{}输出一行,代表$S(n)$从第$k$个位置起的10个字符。如果第$k$个位置起字符数不足10个,则输出到$S(n)$末尾的所有字符。

\divsel{
    \exmpv{A01-sample}
}{
    \exmpv{B01-sample}
    \exmpv{B02-sample}
}

\end{Problem}
