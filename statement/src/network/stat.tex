\addtocounter{ProblemNo}{1}
\renewcommand{\ProblemName}{神经网络}
\begin{center}
\huge{\Alph{ProblemNo}. \textbf{\ProblemName}} \\ [0.8cm]
\large{\textit{时间限制:} [TBD] \textit{秒}} \\ [1cm]
\end{center}

终于,人类的神经网络模型被神秘博士彻底破译。抽象地说,人类的神经网络有三种节点:

\begin{itemize}
\item 输入节点 (I)
\item 输出节点 (O)
\item 隐藏节点 (H)
\end{itemize}

其中节点和节点之间可能存在神经连接,每个神经连接都有其强度 $w$,神经连接构成一个有向图。

JYY无意中获得了一个JitBrains公司赠送的脑电波干扰器。干扰器有一个旋钮,可以设置发射功率 $k$。一旦干扰器启动,对于强度为 $w$ 的神经连接 $u \to v$ ,如果 $w \le k$,则该神经连接将被切断,从而可以降低一定范围内人类的智商。

NJUPC终于要到了。为了防止咖啡鸡屠榜,JYY 打算悄悄把脑电波干扰器安装在咖啡鸡的电脑附近,以便在比赛开始后降低他们的智商。但因为干扰器有一定的作用范围,所以过大的发射功率可能会暴露 JYY 的行为,因此善良的 JYY 希望你帮助他设置最佳的脑电波干扰器的发射功率 $k$:

\begin{enumerate}
\item 把咖啡鸡的智商降低到零,即删除图中强度 $w \le k$ 的神经连接后,对于\textbf{任意}一个输入节点和\textbf{任意}一个输出节点,它们之间都不存在神经连接连通的路径;
\item 对其他选手的影响最小,即你需要找到满足条件 1 的 $k$ 的最小值。
\end{enumerate}

\subsection*{输入格式}

输入数据描述了咖啡鸡的神经网络模型。

输入数据第一行$1\le m \le 10^5$表示神经连接的数量。接下来 $m$ 行,每行由空格分开的三部分组成,形如:\texttt{u -> v (w)},其中:

\begin{itemize}
\item \texttt{u}和\texttt{v}是一个节点的描述,用\texttt{I[x]}, \texttt{O[x]}, \texttt{H[x]}分别表示输入、输出和隐藏节点 $(1 \le x \le 10^5)$;
\item 整数\texttt{w}是神经连接的强度 $(0\le w\le 10^6)$。
\end{itemize}

注意,同一对神经元之间可能有多个神经连接。

\subsection*{输出格式}

输出一个整数,表示答案。

\setcounter{ExampleNo}{0}

\exmpv{01-sample}
\exmpv{02-sample}

\clearpage

\ifodd\value{page}
\else
    \vspace*{\fill}
    \begin{center}
    \textbf{\Large 本页无正文}
    \end{center}
    \vspace*{\fill}
    \clearpage
\fi

