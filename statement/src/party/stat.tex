\begin{Problem}{迎新晚会}{\divsel{7}{5}}

计算机科学与技术系一年一度的迎新晚会马上就要开始了!今年,报名参加迎新晚会的同学格外的多,而每位报名的同学又可以选择表演一个歌舞类节目,或是表演一个语言类节目。这可愁坏了迎新晚会的总导演:该怎样安排节目,既可以使得每位报名的同学都能得到上台表演的机会,又能使观众尽量满意?

为了解决这一难题,总导演提出了以下模型:每位报名同学都有两个属性$x, y$,分别是他所表演的歌舞类节目和语言类节目的精彩程度。观众对歌舞类节目的满意度为所有歌舞类节目精彩程度的最大值;同样,观众对语言类节目的满意度为所有语言类节目精彩程度的最大值。而观众对歌舞类节目、语言类节目的满意度之差的绝对值越小,观众对整场迎新晚会的满意度越高。也就是说,这两类节目的满意度越接近,观众越满意。

数学不好的总导演并不会求最优的节目安排方案,使得观众对整场迎新晚会最为满意。具体来说,他希望你找到一种节目安排方案,满足以下条件:

\begin{itemize}
\item 每位报名同学要么表演一个歌舞类节目,要么表演一个语言类节目;
\item 至少有一位同学表演歌舞类节目,也至少有一位同学表演语言类节目;
\item 在满足上述前提的条件下,观众对两类节目满意度之差的绝对值要尽可能小。
\end{itemize}

你需要求出观众对两类节目满意度之差的绝对值的最小值。

\subsection*{输入格式}

第一行包含一个整数$n$ $(2 \leq n \leq \divsel{3 \times 10^5}{10^4})$,表示报名参加迎新晚会的同学数量。

接下来$n$行,每行包含两个数字$x, y$ $(0 \leq x, y \leq \divsel{10^{18}}{10^9})$,表示一位同学所表演的歌舞类节目和语言类节目的精彩程度。

\subsection*{输出格式}

输出一个整数,表示答案。

\exmpv{001-sample}
\exmpv{002-sample}

\end{Problem}
