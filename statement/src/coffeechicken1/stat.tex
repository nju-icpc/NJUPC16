\addtocounter{ProblemNo}{1}
\renewcommand{\ProblemName}{Coffee or Chicken}
\begin{center}
\huge{\Alph{ProblemNo}. \textbf{\ProblemName}} \\ [0.8cm]
\large{\textit{时间限制:} 1 \textit{秒}} \\ [1cm]
\end{center}

JYY创造了一种“咖啡鸡字符串”,类似于“斐波那契例汤”——第$n$天的汤是由第$n-2$天和第$n-1$天的汤混合而成的:

\begin{itemize}
  \item $S(1) = \texttt{"COFFEE"}$;
  \item $S(2) = \texttt{"CHICKEN"}$;
  \item $S(n) = S(n - 2) :: S(n - 1)$,其中“$::$”是字符串的拼接运算符。
\end{itemize}

随着$n$的增长,$S(n)$的长度很快就失控了,JYY的小霸王学习机没有足够的内存存储$S(n)$。JYY希望你帮助已经被降智的他求出$S(n)$从第$k$个位置开始的连续10个字符。

\subsection*{输入格式}

输入数据第一行包含两个整数 $n$ ($1\le n\le 100$), $k$ ($1\le k\le 10^9$)。对于普及组,$1\le k\le 10^6$。

\subsection*{输出格式}

输出一行,代表$S(n)$从第$k$个位置起的10个字符。如果第$k$个位置起字符数不足10个,则输出到$S(n)$末尾的所有字符。

\setcounter{ExampleNo}{0}

\exmpv{01-sample}
\exmpv{02-sample}

\clearpage

\ifodd\value{page}
\else
    \vspace*{\fill}
    \begin{center}
    \textbf{\Large 本页无正文}
    \end{center}
    \vspace*{\fill}
    \clearpage
\fi

