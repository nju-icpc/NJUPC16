\addtocounter{ProblemNo}{1}
\renewcommand{\ProblemName}{NEMU}
\begin{center}
\huge{\Alph{ProblemNo}. \textbf{\ProblemName}} \\ [0.8cm]
\large{\textit{时间限制:} 1 \textit{秒}} \\ [1cm]
\end{center}

NEMU (NJU EMUlator) 是JYY博士最新研发的s86架构模拟器。s86是一种栈式计算机体系结构,其机器指令只能对栈顶元素进行操作。s86的计算模型包括一个栈和一段有限长的程序。栈中的每个元素均为64位无符号整数类型;程序则由下表中的指令构成,且必须以\texttt{end}指令结束。s86机器运行时,首先会将栈初始化为空,随后将会依次执行程序中的每条指令,直到执行完最后一条\texttt{end}指令后,机器会输出栈顶元素并停机。

\begin{table}[htbp]
\centering
\begin{tabular}{ccc}
\hline
助记符 & 功能 & 限制  \\ \hline
\texttt{p1} & 将常数1压入栈中 & 无 \\ 
\texttt{dup} & 将栈顶元素重复一份后压入栈中 & 当前栈必须非空 \\
\texttt{pop} & 弹出栈顶元素 & 当前栈必须非空 \\
\texttt{swap} & 交换栈顶两个元素 & 当前栈的大小至少是2 \\ 
\texttt{add} $x$ & 栈顶元素加上栈中第$x$个元素 & 当前栈的大小必须大于$x$ \\ 
\texttt{sub} $x$ & 栈顶元素减去栈中第$x$个元素 & 当前栈的大小必须大于$x$ \\
\texttt{mul} $x$ & 栈顶元素乘以栈中第$x$个元素 & 当前栈的大小必须大于$x$ \\
\texttt{end} & 输出栈顶元素并停止程序执行 & 当前栈必须非空,且必须是程序的最后一条指令 \\ \hline
\end{tabular}
\caption{s86架构的指令表}
\end{table}

其中,栈中第$x$个元素表示栈顶向下第$x$个元素。栈顶元素本身是第0个元素。

需要注意的是,s86中所有算术运算指令(\texttt{add}, \texttt{sub}, \texttt{mul})的结果都需要对$2^{64}$取模,即当算术运算的结果为$X$时,s86指令的运算结果为$X'$ $(0 \leq X' < 2^{64})$,并且$X - X'$是$2^{64}$的倍数。可以证明,对于任意整数$X$,这样的$X'$都是唯一存在的。

现在,JYY已经完成了NEMU的开发。为了测试NEMU的正确性,JYY想请你编写一段s86程序,使得机器执行结束后,能够输出给定整数$N$。

\subsection*{输入格式}

输入一个整数$N$ $(0 \leq N < 2^{64})$。

\subsection*{输出格式}

输出一段程序,程序执行结束后,栈顶元素为$N$。

对于公开组,程序最多包含200条指令;对于提高组,程序最多包含60条指令。注意,如果你的程序在执行过程中发生违反指令表中相关限制的行为,你的答案将会被判定为错误。

如果有多解,输出任意一解。保证对于所有输入数据都有解。

\setcounter{ExampleNo}{0}

\exmpv{001-sample}
\exmpv{002-sample}
\exmpv{003-sample}
\exmpv{004-sample}

\subsection*{提示}

$2^{64} = 18 446 744 073 709 551 616$

在C++中,可以使用\verb|std::uint64_t|(在\texttt{cstdint}头文件内)类型实现模$2^{64}$运算。

\clearpage

\ifodd\value{page}
\else
    \vspace*{\fill}
    \begin{center}
    \textbf{\Large 本页无正文}
    \end{center}
    \vspace*{\fill}
    \clearpage
\fi

