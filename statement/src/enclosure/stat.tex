\addtocounter{ProblemNo}{1}
\renewcommand{\ProblemName}{小裙子}
\begin{center}
\huge{\Alph{ProblemNo}. \textbf{\ProblemName}} \\ [0.8cm]
\large{\textit{时间限制:} 1 \textit{秒}} \\ [1cm] 
\end{center}

今天,RoundGod来到了商场,并发现了很多的小裙子!

于是RoundGod开心的跑了过去,仔细的挑起了小裙子。

他发现:小裙子上有两个矩形图案装饰,而小裙子的展开图可以看作是一个无限大的二维坐标系,他想要知道这两个矩形能把小裙子分成多少个连通区域。

\subsection*{输入格式}

输入包括两行,分别描述这两个矩形。

输入每行包括四个整数$x_1, y_1, x_2, y_2$ $(x_1 < x_2, y_1 < y_2)$,分别表示这个矩形的左下角横坐标、纵坐标和右上角横坐标、纵坐标。

保证第一个矩形左下角、右上角,第二个矩形的左下角、右上角的横坐标互不相同,这四个点的纵坐标也互不相同。

\subsection*{输出格式}

输出一行,包含一个整数,表示这两个矩形将平面划分为连通区域的个数。

\setcounter{ExampleNo}{0}

\exmpv{01-sample}
\exmpv{02-sample}
\exmpv{03-sample}

\clearpage

\ifodd\value{page}
\else
    \vspace*{\fill}
    \begin{center}
    \textbf{\Large 本页无正文}
    \end{center}
    \vspace*{\fill}
    \clearpage
\fi

